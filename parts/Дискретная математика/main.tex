\part{Дискретная математика}

\chapter{Теория множеств}

\section{Множества}

\subsection{Счётные множества}

Счётными множествами являются:
\begin{itemize}
	\item Множество $N$ натуральных чисел;
	\item Множество $Z$ целых чисел;
	\item Множество $Q$ рациональных чисел.
\end{itemize}

\begin{theorem}[Теорема Контора]
	Множество вещественных чисел из интервала $(0; 1)$ не является счётным.

	\begin{proof}[Доказательство]
		Предположим, что $(0;1)$ --- счётное множество. Тогда его элементы можно
		занумеровать. Будем записывать вещественные числа бесконечными десятичными
		дробями, если нужно приписывая нули. Тогда перечисление всех чисел из
		$(0;1)$ можно представить бесконечной таблицей:

		\[
			\begin{matrix}
				a_1 = 0,    & \alpha_{11} & \alpha_{12} & \alpha_{13} & \ldots
				\alpha_{1n} & \ldots                                           \\
				a_2 = 0,    & \alpha_{21} & \alpha_{22} & \alpha_{23} & \ldots
				\alpha_{2n} & \ldots                                           \\
				a_3 = 0,    & \alpha_{31} & \alpha_{32} & \alpha_{33} & \ldots
				\alpha_{3n} & \ldots                                           \\
				a_4 = 0,    & \alpha_{41} & \alpha_{42} & \alpha_{43} & \ldots
				\alpha_{4n} & \ldots                                           \\
			\end{matrix}
		\]

		Здесь первый индекс показывает номер числа, а второй --- номер
		разряда. Запишем число
		\[
			b = 0, \beta_1 \beta_2 \beta_3 \dots \beta_n \dots
			,\]
		где $\beta_1 \neq \alpha_{11}, \beta_2 \neq \alpha_{22}, \dots,
			\beta_n \neq \alpha_{nn}, \dots$. Число $\beta$ принадлежит
		интервалу $(0;1)$ и отличается по крайней мере одним десятичным
		знаком от любого числа из таблицы. Значит, вопреки предположению,
		число $\beta$ не входит в перечисление. Следовательно и множество
		чисел $(0;1)$ не счётно.
	\end{proof}
\end{theorem}

Множество $A$, эквивалентное $(0;1)$ называется множеством мощности
континуум\index{континуум}\footnote{То, что можно продолжить, сплошное
	множество}. Любой интервал $(a;b)$, любое множество $R$, любое $n$--мерное
множество $R^n$ содержит континуум.

\chapter{Комбинаторика}

\begin{definition}[Комбинаторика]
	Это раздел математики, в котором изучаются способы подсчёта
	расположений (комбинаций) объектов проивзодного множества.
\end{definition}

\section{Основные принципы комбинаторики}

\begin{definition}[Принцип умножения]
	Если какой-то объект $A$ можно выбрать $n$ различными способоами, а
	объект $B$ --- $m$ разлчиными способами. Тогда выбор из $A$ и $B$
	можно реализовать $m \cdot n$ способами.

	Если мы мысленно можем спросить ``Сколько способов сделать это И это
	И это?'' (союз ``И''), то это точно принцип умножения.
\end{definition}

\begin{definition}[Принцип сложения]
	Если какой-то объект $A$ можно выбрать $n$ различными способоами, а
	объект $B$ --- $m$ разлчиными способами. Тогда выбор из $A$ или $B$

	Если мы мысленно можем спросить ``Сколько способов сделать это ИЛИ
	это ИЛИ это?'' (союз ``ИЛИ''), то это точно принцип сложения.можно
	реализовать $m + n$ способами.
\end{definition}

\subsection{Перестановка, сочетания, размещения}

\begin{definition}[Перестановка]
	Перестановка из $n$ элементов называется их расположение в
	некотором порядке. Число всех перестановок обозначают $P_n$.
\end{definition}

\chapter{Математическая логика}

Математическая логика --- логика, рассматриваемая с точки зрения математики.

\section{Теория высказываний}

\begin{definition}
	Высказыванием называется предложение либо истинное, либо ложное. Например:
	\begin{itemize}
		\item ``Снег белый'' --- истина;
		\item ``2 \cdot 2 = 5'' --- ложь.
	\end{itemize}

	Произвольные высказывания мы будем обозначать заглавными буквами:
	$A, B, Q, P, \dots$. Они называются \emph{пропозиционными
		переменными}\index{пропозиционная~переменная}.
\end{definition}

Введём операции над высказываниями:
\begin{itemize}
	\item Отрицание $\neg P$, которое понимается, как ``Не $P$'';
	\item Конъюнкция $P \And Q$, соответствует союзы "И". Понимается
	      как $P$ и $Q$;
	\item  Дизъюнкция $P \vee Q$, соответствует союзы "ИЛИ".
	      Понимается, как $P$ или $Q$;
	\item Импликация $P \supset Q$, соответствует союзы "если, то".
	      Понимается, как "если $P$, то $Q$".
	\item Эквивалентность $P \sim Q$, соответствует союзу "тогда и
	      только тогда (титт)". Понимается, как " $P$ тогда и только тогда,
	      когда $Q$".
\end{itemize}

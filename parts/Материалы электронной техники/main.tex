\part[Материалы электронной техники]{
Материалы электронной техники \\
{\Large Демаков Юрий Павлович}
}

\chapter{2025-02-06. }

\subsubsection{Параметры материалов электротехники}

Для проводниковых материалов действует закон Ома в дифференциальной форме
\[
	j = \sigma E
	,\]
где $j$ --- плотность электрического тока, \unit{\ampere\per\square\meter}; $E$
--- напряжённость вощбуждающего электрического поля, \unit{\volt\per\meter};
$\sigma = \frac{l}{\rho}$ --- удельная проводимость проводника,
\unit{\per\ohm\per\meter}; $\rho$ --- удельное сопротивление проводника,
\unit{\ohm\meter}.

Для диэлектрическоих материалов выполняется следующее соотношение:
\[
	D = \varepsilon_0 \varepsilon E
	,\]
где $D$ --- электрическая индукция, \unit{\kelvin\per\square\meter};
$\varepsilon_0 = \qty{8.854e-12}{\farad\per\meter}$ --- электрическая
постоянная; $\varepsilon$ --- диэлектрическая проницаемость.

\subsubsection{Коэфициенты нестабильности}

Коэффициент настабильности представляет собой относительную величину отклонения
параметра при изменении или (в ряде случаев) при фиксированном значении
воздействия. Например, \emph{температурный коэффициент} параметра $y$
(обозначается $\alpha_{y, T}$) характеризует относительное изменение параметра
$y$ при изменении температуры $T$ окружающей среды на один градус:
\[
	\alpha_{y, T} = \frac{y_2 - y_1}{y} \frac{1}{T_2 - T_1} = \frac{\Delta y}{y}
	\frac{1}{\Delta T}, \unit{\per\kelvin}
	,\]
где $\alpha_{y,T}$ --- коэффициент, характеризующий относительное изменение
параметра $y$ под влиянием дестабилизирующего фактора --- температуры; $\Delta y
	= y_2 - y_1$ --- изменение параметра в диапозоне значений от $y_1$ до $y_2$ при
изменении температуры $T$ на величину $\Delta T = T_2 - T_1$.

\subsubsection{Конденсация вещества}

При сближении атомов до расстояния нескольких долей нанометра между ними
появляются силы взаимодействия. Если эти силы являются силами притяжения, то
атомы могут соединиться с выделением энергии, образуя устойчивые химичекие
соединения. Число атомов, содержащихся в объёме вещества массой $m$ равно
\[
	N = N_A \frac{m}{M}
	,\]
где $N_A = \qty{6.02e26}{\per\kilo\mol}$ --- число Авагадро; $M$ --- атоманая
(молярная) масса вещества.

\subsubsection{Примитивная элементарная ячейка}

Кристалическая структура представляет решётку, построенную на трёх координатных
осях $x, y, z$, расположенных в общем случае под углами $\alpha, \beta, \gamma$.
Периоды трансляции атомов по осям (параметры решётки) равны соответственно $a,
	b, c$. Элементарная ячейка кристалла --- это паралепипед, построенный на
векторах трансяции $a, b, c$. Такая элементарная ячейка называется
\emph{примитивной}.

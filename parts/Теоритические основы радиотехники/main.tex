\part[Теоритические основы радиотехники]{
Теоритические основы радиотехники \\
{\Large Хворенков Владимир Викторович}
}

\chapter{2025-02-05. Радиотехнические цепи и синалы}

\section{Радиотехнические сигналы}

\begin{definition}[Сигнал]
	Некоторое явление, которое переносит информацию; процесс изменения во времени
	физического состояния некоторого объекта, который служит для отображения,
	передачи и регистрации информации.
\end{definition}

\begin{definition}[Кластер]
	Ограниченное подмножество некоторого множества.
\end{definition}

Для записи сигнала используется Математическая модель
\[ y = ax \qquad y = \frac{a}{b} \]
$S(t)$ --- функция, изменяющаяся во времени.

\section{Классификация сигналов}

\begin{enumerate}
	\item Одномерные $S(t) = U_m \cos (\omega_0 t)$ и многомерные сигналы;
	\item Детерминированный (подлежит вычислению в любой момент времени) и
	      случайное (стахастический);
	\item Импульсные сигналы --- в пределах, ограниченных конечным отрезком
	      времени (видеоимпульсы и радиоимпульсы).
\end{enumerate}

\begin{figure}[!htbp]
	\centering
	\begin{tikzpicture}
		\draw[->] (0, 0) -- (7, 0) node[right] {$t$};
		\draw[->] (0, 0) -- (0, 4) node[above] {$S(t)$};

		\draw (0, 0) to[in=180, out=0] (2, 3.2) to[in=180, out=0] (3, 2.5)
		to[in=180, out=0] (4, 3) to[in=180, out=0] (6, 0);

		\draw[dashed] (0, 3.2) node[left] {$S_m$} -- (6, 3.2);
		\draw[dashed] (0, 2.26) node[left] {$\frac{S_m}{\sqrt{2}}$} -- (6, 2.26);
		\draw[dashed] (1.13, 2.26) -- ++(0, -2.26);
		\draw[dashed] (4.83, 2.26) -- ++(0, -2.26);
		\node[below] at (0.565, 0) {$t_\varphi^{(t)}$};
		\node[below] at (3, 0) {$t_u$};
		\node[below] at (5.415, 0) {$t_\varphi^{(t)}$};
	\end{tikzpicture}
	\caption{Пример видеосигнала}\label{fig:video-signal}
\end{figure}

\part{Правоведение}

\chapter{Норма права: понятие, структура, классификация}

\begin{definition}[Норма права]
  Общие обязательные формально определённые правила поведения,
  установленные и обеспеченные государством, направленные на
  урегулирование общественных отношений.
\end{definition}

Норма права характеризуется следующими признаками:

\begin{enumerate}
  \item Она является разновидностью социальных норм;
  \item Она носит общеобязательный характер;
  \item Устанавливаются государством либо с его санкциями
    негосударственными субъектами правотворчества;
  \item Обладает (по крайней мере \emph{должна} обладать) формальной
    определённостью;
  \item Её реализация обеспечивается государственными гарантиями и принуждением;
  \item Должна отвечать критерию системности;
  \item Регулирует общественные отношения;
  \item Обладает волевым характером;
  \item Обладает своей структурой.
\end{enumerate}

Структура норм права --- это выводимое логическим путём упорядоченное единством
взаимосвязь её составных элементов, а именно
гипотезы~(\ref{def:hypoth}), диспозиции~(\ref{def:disposition}),
санкций~(\ref{def:sanction}).

\begin{definition}[Гипотеза]\label{def:hypoth}
  Указывает на наличие или отсутствии жизни жизненных обстоятельств,
  условий, при которых реализуются норма права.
\end{definition}

\begin{definition}[Диспозиция]\label{def:disposition}
  Содержит все само правил поведения, согласно которого должны
  действовать участники правоотношений.
\end{definition}

\begin{definition}[Санкция]\label{def:sanction}
  Указывает, на неблагоприятные последствия, которые наступают в
  результате нарушения диспозиции.
\end{definition}

Выделяют следующую классификацию норм права:

\begin{itemize}
  \item По предмету правового регулирования, то есть по отраслевой
    принадлежности:
    \begin{itemize}
      \item нормы конституционного права,
      \item нормы гражданского права,
      \item нормы уголовного права,
      \item нормы гражданско-процессуального права,
      \item норма уголовно-процессуального права,
      \item нормы административного права,
      \item нормы административного судопроизводства,
      \item нормы трудового права,
      \item норма семейного права,
      \item нормы финансового права,
      \item нормы хозяйственного права,
      \item нормы страхового права,
      \item нормы таможенного права,
      \item нормы градостроительного права,
      \item нормы Жевнова права,
      \item нормы нового права,
      \item нормы воздушного права,
      \item нормы водного права,
      \item нормы торгового мореплавания,
      \item нормы налогового права,
      \item норма бюджетного права,
      \item нормы арбитражно-процессуального права
    \end{itemize}
  \item по их юридической силе:
    \begin{enumerate}
      \item Нормы законов
        \begin{itemize}
          \item федерально конституционные,
          \item федеральные
            \begin{itemize}
              \item уголовный кодекс,
              \item арбитражный процессуальный кодекс,
              \item уголовно-процессуальный кодекс,
              \item гражданско-процессуальный кодекс,
              \item семейный кодекс,
              \item кодекс Российской Федерации об административных
                правонарушениях.
            \end{itemize}
          \item Региональная
            \begin{itemize}
              \item Конституция удмуртской республики,
              \item устав города Москвы,
              \item кодекс об административных правонарушениях
                республики Татарстан,
              \item Конституция республики Башкортостан,
              \item кодекс об административных правонарушениях города Москвы,
              \item устав свердловской области,
              \item Конституция республики Ингушетии.
            \end{itemize}
        \end{itemize}
      \item Под законные акты. Например, следующие разновидности:
        \begin{itemize}
          \item УК за президента Российской Федерации,
          \item постановление Роспотребнадзора Российской Федерации,
          \item постановление министерства здравоохранения,
          \item постановление в федеральное медикологической агентства,
          \item указы головы удмуртской республики,
          \item указы головы мапу РГИ Скова района,
          \item указа главы Ярского района,
          \item указа головы города Перми,
          \item постановление правительства республики Татарстан,
          \item указы головы города Можги.
        \end{itemize}
    \end{enumerate}
\end{itemize}

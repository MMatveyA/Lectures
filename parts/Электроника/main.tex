\part[Электроника]{
Электроника \\
{\Large Сидорина Вера Анатольевна}
}

\chapter{2025-02-04. Основные определения}

\begin{definition}[Электроника]
	Отрасль науки и техники, изучаящая законы взаимодействия электронов и других
	заряженных частиц с электромагнитными полями и разрабатывающая  методы
	создания электронных приборов, в которых это взаимодействией используется для
	преобразования электромагнитной энергии с целью передачи, обработки и хранения
	информации, автоматизации производственных процессов, создания аппаратуры,
	устройств и средств контроля, измерения и управления.
\end{definition}

Основные направления:
\begin{itemize}
	\item Вакуумное;
	\item Твердотельное;
	\item Квантовая;
\end{itemize}

Основой лекций будет изуччение полупроводниковых приборов и процессов в них.

\section{Элементы электронных схем}

\emph{Полупроводниками} называют материалы, удельное сопротивление которых при
комнатной температуре (\qtyrange{25}{27}{\degreeCelsius}) находятся в пределах
от \qtyrange{1e-5}{1e10}{\ohm\meter} и занимающими промежуточное положение между
металлами и диэлектриками. Объяснение этому явлению даёт \emph{Теория
	электропроводности}, согласно которой атому вещества состоит из ядра окружённого
облаком электронов.

\emph{Электроны} находятся в движении на некотором расстоянии от ядра в пределах
слоёв (оболочек), определяемых их энергией. Каждому их этих слоёв можно
поставить в соответствие определённый энергетический уровень электрона, причём
чем дальше электрон находится от ядра, тем выше его энергетический уровень.
Совокупность энергетических уровней образует энергетический спектр. Если
электрон переходит с одного энергетического уровня на другой, то происходит либо
выделение, либо поглощение энергии, причём это делается пропорциями ---
квантами.

В структуре атомов можно выделить оболочки, которые полностью заняты электронами
(внтуренние оболочки) и незаполненные оболочки (внешние). Электроны внешних
оболочек слабее связаны с ядром и легче вступают во взаимодействие с другими
атомами. Электроны внешних оболочек называют валетными.

Для полупроводниковых материалов характерно \emph{кристалическое строение} при
котором между атомами возникают так называемые ковалентные связи за счёт
присвоения соседних валентных электронов. Это наглядно можно показать на плоской
модели кристалической решётки, например для четырёхвалетного германия Ge.

Атомы связанф между собой, то есть их электроны находятся на взаимозависимых
энергетических (расщеплённых) уровнях.

При этом на каждом уровне не более двух электронов. Совокупность энергетических
уровней, на которых могут находиться электроны, называют \emph{разрешёнными
	зонами}. Между ними будут в этом случае распологаться \emph{запрещённые зоны},
то есть энергитические уровни, на которых электроны находиться не могут. В
соответствии с зонной теорией по отношению к энергетическим уровням электронов
различают \emph{валетную зону}, \emph{запрещённую зону}, \emph{зону
	проводимости}. В такой интерпретации можно более определённо разделить все
вещества на три большие группы металлы, полупроводники, диэлектрики.

\begin{definition}[Зона проводимости]
	Совокупность расщеплённых энергетических уровней, на которые может переходить
	электрон в процессе взаимодействия атомов или воздействия на атом, например,
	при нагреве, облучении. У полупроводников при некотором значении температуры
	часть электронов приобретает энергию тепла и оказывается в зоне проводимости.
\end{definition}

Эти электроны делаюь полупроводники электропроводными. Если электрон покидает
валентную зону, то образуется свободный энергетический уровень, как бы вакантное
место (состояние), которое назвали <<дыркой>>. Валентные электроны соседних
атомов могут переходить на эти свободные уровни, при этом создают дырки в других
атомах.

Такое \emph{перемщение} электронов рассматривается как движение положительных
зарядов --- <<дырок>>. Соответственно электропроводность, обусловленная
движением электронов называется \emph{электронной}, а движение дырок ---
<<дырочной>>. У абсолютно чистого и однородного вещества свободные электроны и
дырки образуются попарно.

Техническое применение получили так называемые примесные полупроводники, в
которых, в зависимости от рода введённой примеси, преобладает либо электронная,
либо дырочная проводимость. В зависимости от типа проводимости (основных
носителей заряда) полупроводники подразделяются на полупроводники $p$--типа
(дырочного пути) и $n$--типа (электронного типа).

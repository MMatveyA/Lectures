\part[Электроника]{
Электроника \\
{\Large Сидорина Вера Анатольевна}
}

\chapter{2025-02-04. Основные определения}

\begin{definition}[Электроника]
	Отрасль науки и техники, изучаящая законы взаимодействия электронов и других
	заряженных частиц с электромагнитными полями и разрабатывающая  методы
	создания электронных приборов, в которых это взаимодействией используется для
	преобразования электромагнитной энергии с целью передачи, обработки и хранения
	информации, автоматизации производственных процессов, создания аппаратуры,
	устройств и средств контроля, измерения и управления.
\end{definition}

Основные направления:
\begin{itemize}
	\item Вакуумное;
	\item Твердотельное;
	\item Квантовая;
\end{itemize}

Основой лекций будет изуччение полупроводниковых приборов и процессов в них.

\section{Элементы электронных схем}

\emph{Полупроводниками} называют материалы, удельное сопротивление которых при
комнатной температуре (\qtyrange{25}{27}{\degreeCelsius}) находятся в пределах
от \qtyrange{1e-5}{1e10}{\ohm\meter} и занимающими промежуточное положение между
металлами и диэлектриками. Объяснение этому явлению даёт \emph{Теория
	электропроводности}, согласно которой атому вещества состоит из ядра окружённого
облаком электронов.

\emph{Электроны} находятся в движении на некотором расстоянии от ядра в пределах
слоёв (оболочек), определяемых их энергией. Каждому их этих слоёв можно
поставить в соответствие определённый энергетический уровень электрона, причём
чем дальше электрон находится от ядра, тем выше его энергетический уровень.
Совокупность энергетических уровней образует энергетический спектр. Если
электрон переходит с одного энергетического уровня на другой, то происходит либо
выделение, либо поглощение энергии, причём это делается пропорциями ---
квантами.

В структуре атомов можно выделить оболочки, которые полностью заняты электронами
(внтуренние оболочки) и незаполненные оболочки (внешние). Электроны внешних
оболочек слабее связаны с ядром и легче вступают во взаимодействие с другими
атомами. Электроны внешних оболочек называют валетными.

Для полупроводниковых материалов характерно \emph{кристалическое строение} при
котором между атомами возникают так называемые ковалентные связи за счёт
присвоения соседних валентных электронов. Это наглядно можно показать на плоской
модели кристалической решётки, например для четырёхвалетного германия Ge.

Атомы связанф между собой, то есть их электроны находятся на взаимозависимых
энергетических (расщеплённых) уровнях.

При этом на каждом уровне не более двух электронов. Совокупность энергетических
уровней, на которых могут находиться электроны, называют \emph{разрешёнными
	зонами}. Между ними будут в этом случае распологаться \emph{запрещённые зоны},
то есть энергитические уровни, на которых электроны находиться не могут. В
соответствии с зонной теорией по отношению к энергетическим уровням электронов
различают \emph{валетную зону}, \emph{запрещённую зону}, \emph{зону
	проводимости}. В такой интерпретации можно более определённо разделить все
вещества на три большие группы металлы, полупроводники, диэлектрики.

\begin{definition}[Зона проводимости]
	Совокупность расщеплённых энергетических уровней, на которые может переходить
	электрон в процессе взаимодействия атомов или воздействия на атом, например,
	при нагреве, облучении. У полупроводников при некотором значении температуры
	часть электронов приобретает энергию тепла и оказывается в зоне проводимости.
\end{definition}

Эти электроны делаюь полупроводники электропроводными. Если электрон покидает
валентную зону, то образуется свободный энергетический уровень, как бы вакантное
место (состояние), которое назвали <<дыркой>>. Валентные электроны соседних
атомов могут переходить на эти свободные уровни, при этом создают дырки в других
атомах.

Такое \emph{перемщение} электронов рассматривается как движение положительных
зарядов --- <<дырок>>. Соответственно электропроводность, обусловленная
движением электронов называется \emph{электронной}, а движение дырок ---
<<дырочной>>. У абсолютно чистого и однородного вещества свободные электроны и
дырки образуются попарно.

Техническое применение получили так называемые примесные полупроводники, в
которых, в зависимости от рода введённой примеси, преобладает либо электронная,
либо дырочная проводимость. В зависимости от типа проводимости (основных
носителей заряда) полупроводники подразделяются на полупроводники $p$--типа
(дырочного пути) и $n$--типа (электронного типа).

% \section{Параметры диода}

\subsection{Полупроводниковые диоды}

\begin{definition}[Полупроводниковый диод]
	Это электропреобразовательный полупроводниковый прибор с одним выпрямляющим
	электрическим переходом, имеющим два вывода.
\end{definition}

Буквами $p$ и $n$ обозначены слои полупроводника с проводимостями соответственно
$p$ и $n$--типа. В контактирующих слоях полупроводника (область
$p$--$n$--перехода) имеет место \emph{диффузия} дырок из слоя $p$ в слой $n$,
причиной которой является то, что их концетрация в слое $p$ значительно больше
их концетрации в слое $n$.

В итоге в приграничных областях слоя $p$ и слоя $n$ возникает так называемый
обеднённый слой, в котором мала концетрация подвжиных носителей заряда.

\subsubsection{Основные характеристики}

Основные характеристики полупроводникового диода представляется его
вольт--амперной характеристикой (ВАХ).

\begin{definition}[Вольт--амперная характеристика]
	Зависимость тока $I$, протекающего через диод, от напряжения $U$, приложенного
	к диоду.
\end{definition}

\subsubsection{Параметры диода}

Основными параметрами диода являются:
\begin{itemize}
	\item Прямое напржение $U_\text{пр}$ --- значение постоянного прямоо
	      напряжения при заданном токе $I_\text{пр}$;
	\item Обратный ток $I_\text{обр}$ --- значение постоянного тока, протекающего
	      через диод в обратном направлении при заданном обратном напряжении
	      $U_\text{обр}$;
	\item Сопротивление диода в прямом напряжении $R_\text{пр} =
		      \sfrac{U_\text{пр}}{I_\text{пр}}$, величина которог зависит от $U_\text{пр}$
	      и составляет единицы и десятки омов;
	\item  Дифференциальное сопротивление $R_\text{диф} = \sfrac{\Delta
			      U_\text{пр}}{\Delta I_\text{пр}}$, величина которого зависит от
	      $I_\text{пр}$ и $U_\text{пр}$ и может составлять единицы омов. Его можно
	      определить графически в любой точке ВАХ диполя;
	\item Падение напряжения на диоде при некотором значении прямого тока через
	      диод $U_\text{пр}, \unit{\volt}$;
	\item Ëмкость диода при подаче на него обратного напряжения некоторой величины
	      $C, \unit{\pico\farad}$;
	\item Диапозон частот, в котором возможно работа без снижения выпрямленного
	      тока $f_\text{гр}, \unit{\kilo\hertz}$;
	\item Рабочий диапозон тесператур.
\end{itemize}

По конструкции диоды делят на:
\begin{itemize}
	\item \emph{Плоскостные диоды} имеют плоскостной переход, точечный переход
	      создаётся около контакта острия металлической пружинки с ПП кристаллом
	      $n$--типа;
	\item \emph{Точечный диод} имеет малую ёмкость $p$--$n$--перехода, поэтому
	      могут работать в диапозоне высоких частот, но они допускают малые прямые
	      токи (до единицы \unit{\milli\ampere}) и небольшие обратные напряжения (до
	      нескольких десятков \unit{\volt}). Поэтому их применяют в маломощных
	      электронных устройствах промышленной электроники, радиотехники,
	      вычислительной техники, \ldots
\end{itemize}

\subsection{Выпрямительные диоды}

\begin{definition}[Выпрямительные диоды]
	Диод, для которого предельная частота $f$ не превышает величину $f_\text{пр}
		\leq \qty{100}{\kilo\hertz}$.
\end{definition}

\subsubsection{Основные параметры}
Основные параметры:
\begin{itemize}
	\item $I_\text{пр ср}$ --- среднее значение прямого тока, который может
	      длительное время протекать через $p$--$n$--переход. Это значение составляет:
	      \begin{itemize}
		      \item для выпрямительных диодов малой мощности $I_\text{пр ср} =
			            \qty{300}{\milli\ampere}$;
		      \item для выпрямительных диодов средней мощности;
		      \item для выпрямительных диодов большой мощности $I_\text{пр ср} >
			            \qty{10}{\milli\ampere}$;
	      \end{itemize}
	\item $U_\text{пр ср}$ --- среднее значение прямоо напряжения, при
	      заданном $I_\text{пр ср}$.
	\item $U_\text{обр} = U_\text{проб}$ --- напряжение пробоя, его величина
	      сильно зависит от материала диода. При расчётах обычно используют на
	      $\qty{20}{\percent}$.
	\item Диапозон рабочих частот: $\Delta f = f_{\max} - f_{\min}$.
\end{itemize}

\subsubsection{Выбор выпрямительных диодов}

Наибольшее распространение на практике получили кремниевые диоды, так как они
имеют меньший обратный ток $I_\text{обр}$ и большее допустимое обратное
напряжение $U_\text{обр}$.

Несмотря на указанные преимущества кремниевых диодов, в выпрямителях на малые
токи и напряжение лучше использовать германиевые, так как прямая ветвь ВАХ будет
круче, а напряжение отпирания $p$--$n$ перехода в германиевых диодах почти в два
раза меньше соответствующей величины для кремниевых диодов.

\subsection{Стабилитрон}

\begin{definition}[Стабилитрон]
	Полупроводниковый диод, падение обратного напряжения на котором в рабочем
	диапозоне мало зависит от тока, протекающего через него. Таким образом,
	стабилитрон может исопльзоваться для стабилизации напряжения.
\end{definition}

\subsubsection{Основные параметры}

\begin{enumerate}
	\item $U_\text{ст ном}$ --- номинальное напряжение стабилизации. Это
	      напряжение стабилизации при некотором значении $I_\text{ст}$. Диапозон
	      возможных напряжений \qtyrange{2}{1000}{\volt};
	\item $R_\text{Д} = \frac{d U_\text{ст}}{d I_\text{ст}}$ --- динамческое
	      (дифференциальное) сопротивление, представляет собой изменение $U_\text{ст}$
	      при изменении $I_\text{ст}$. Чем меньше $R_I$, тем лучше коэффициент
	      стабилизации. Дифференциальное сопротивление стабилитрона может быть в
	      пределах \qtyrange{0.5}{250}{\ohm}.
	\item $I_{\text{ст } \max}$ --- минимальный ток стабилизации, его величина
	      ограничена нелинейным участком ВАХ стабилитрона.
\end{enumerate}

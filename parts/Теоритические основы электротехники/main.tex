\part{Теоритические основы электротехники}

\chapter{Включение RC цепи на гармоническое напряжение}

Пусть при $t \geq 0$ функция внешнего воздействия $e(t) = U_m
\cos(\omega t + \varphi)$.

Переходной процесс в цепи описывается уравнением по второму закону
Кирхгофа для мгновенных значений величин: $U_R + U_C = e(t)$
$$iR + \frac{1}{C} \int_{-\infty}^t i dt = U_m \sin(\omega t + \psi)$$
С учётом выражения $C = \frac{q}{U}$ или $q = CU$ можно записать
$$RC \frac{dU_c}{dt} + U_C = U_m \cos(\omega t + \psi)$$

Решение этого уравнения:
\[ U_C = U_\text{Cпр} + U_\text{Cсв}, \]
где $U_\text{Cпр}$ и $U_\text{Cсв}$ находятся из уравнений
\[ RC \frac{dU_{Cпр}}{dt} + U_{Cпр} = U_m \cos(\omega t + \psi) \]
\[ RC \frac{d U_{Cсв}}{dt} + U_{Cсв} = 0 \]
Принужденная составляющая находится как напряжение на ёмкости в
установившемся режиме:
\[ U_\text{Cпр} = I_m \frac{1}{mC} \cos(\omega t + \psi - \varphi -
\frac{\pi}{2}) \]
В этом уравнении $I_m = \frac{U_m}{\sqrt{R^2 + \left(\frac{1}{\omega
C}\right)^2}}$ — амплитуда тока, $\varphi = \arctan \left(-\frac{X_C}{R}\right)
= \arctan\left(-\frac{1}{\omega CR}\right)$ --- угол  сдвига фаз  между
напряжением током.

Однородное для свободной составляющей решим записав
характеристическое уравнение:
\[ R Cp + 1 = 0, \]
где $p = -\frac{1}{RC}$, $\tau = RC$ — постоянная времени.

Тогда решение исходного уравнения:
\[ U_C = U_\text{Cсв} = A \cdot e^{-\frac{t}{RC}} \]
Тогда временную зависомость напряжение на ёмкости получим в виде:
\[ U_C = I_m \frac{1}{\omega C} \cos(\omega t + \psi + \varphi -
\frac{\pi}{2}) + A \cdot e^{-\frac{t}{RC}} \]

Постоянная интегрирования определяется их начальных  условий. В
момент коммутации $t = 0$ напряжения на ёмкости  равно нулю и оно
скачком изменяться не может
\[ 0 = I_m \frac{1}{\omega C} \cos(\psi - \varPhi - \frac{\pi}{2}) + A \]

Тогда постоянная интегрирования $A = -I_m \frac{1}{\omega C}
\cos(\psi - \varphi - \frac{\pi}{2})$. Таким образом, напрядение на
ёмкости в переходном напряжении:
\[ U_C = I_m \frac{1}{\omega C} \cos(\omega t + \psi - \varphi -
  \frac{\pi}{2}) - I_m \frac{1}{\omega C} \cos(\psi - \varphi -
\frac{\pi}{2}) \cdot e^{-\frac{t}{RC}} \]

Ток в цепи
\[ i = C \frac{dU_C}{dt} = -I_m \sin(\omega t + \psi - \varphi -
  \frac{\pi}{2}) + I_m \frac{1}{\omega RC}e^{-\frac{t}{RC}} \cdot
\cos(\psi = \varphi - \frac{\pi}{2}) \]

Закон изменения напряжения на активном сопротивлении
\[ U_R = iR = -I_m R \cdot \sin(\omega t + \psi - \varphi) + I_m
  \frac{1}{\omega C} e^{-\frac{t}{RC}} \cdot \cos(\psi - \varphi -
\frac{\pi}{2}) \]

В установившемся режиме, то есть при $t \to \infty$ $U_C =
U_{Cпр}$. В момент времени $t = (0+)$
\[ i(0+) = I_m[\sin \varphi \sin(\varphi - \psi) - \cos \varphi
\cos(\varphi - \psi)] = I_m \cos \psi = \frac{_m}{R} \cos \psi \]
Это говорит о том, что в начальный момент времени  конденсатор как
бы замыкается накоротко и величина тока в цепи полностью зависит
только от $R$.

\chapter{Переходной процесс в цепи rL}

Дифференциальное уравнение для $t\ge0$  имеет вид $n+L\frac{\partial
i}{\partial t} = e(t)$. Его характеристическое уравнение $r+pL=0$,
откуда корень $p_1=-\frac{r}{L}$ и $I_{св}(t) = A_1 e^{p_1t} = A_1
e^{\frac{r}{L}t}$. Обший ток определяется суммой $i(t) = i_{пр}(t) +i_{св}(t)$.
Рассмотрим 3 случая:
\begin{enumerate}
  \item  $e(t) = const$;
  \item  Короткое замыкание цепи $r, L$;
  \item  $e(t) = E_m \cos(\omega t + \psi)$ — то есть гармонический ЭДС.
\end{enumerate}

\section{Включение rL в цепь постоянной ЭДС}

$i_{пр} = \frac{E}{r}$ откуда $i=\frac{E}{r} + Ae^{-\frac{rt}{K}}$.

Постоянная регулирования $A$ находится по НУ\footnote{Нормальных
условиям} $i(0) = i(0-) = 0$, то есть при $t = 0$ уравнение 2 имеет
вид: $0 = \frac{E}{r}-A \to A = -\frac{E}{r}$ и следовательно $i =
\frac{E}{r} (1 - e^{-\frac{rt}{L}}) = I(1 - e^{-\frac{rt}{L}})$.

$\tau = \frac{L}{r}$ — постоянная времени.

\chapter{Собственные процессы в RLC цепи}

Рассмотрим цепь, образованную последовательным соединением активного
сопротивления $R$, индуктивности $L$ и ёмкости $C$ при переменном
воздействии $e(t)$.

Уравнение по второму закону Кирхгофа
\[ Ri + L \frac{di}{dt} + \frac{1}{C} \int i dt = e(t) \]

Продифференцировав обе части равенства по $t$ и разделив их на $L$,
получим дифференциальное уравнение для тока:
\[ \frac{d^2i}{dt^2} + \frac{R}{L} \cdot \frac{di}{dt} + \frac{1}{LC}
\cdot i = \frac{1}{L} \cdot \frac{de}{dt} \]
Если функция внешнего воздействия $e(t) = 0$ при любых $t \geq 0$
(режим короткого замыкания цепи), то уравнение становится однородным:
\[ \frac{d^2i}{dt^2} + \frac{R}{L} \cdot \frac{di}{dt} + \frac{1}{LC}
  \cdot i = 0$$ или $$\frac{d^2i}{dt^2} + 2\alpha \cdot \frac{di}{dt} +
\omega^2_0 \cdot i = 0 \]

Общее решение его имеет вид
\[ i(t) = i_{св} = Ae^{p_1t} + Be^{p_2t}, \]
где $p_1$ и $p_2$ — корни характеристического уравнения $p^2 + 2
\alpha p + \omega^2_0 = 0$ определяемые равенством $p_{1,2} = -\alpha
\pm \sqrt{a^2 - \omega^2_0}$.

Обозначив $\beta = \sqrt{a^2 - \omega^2_0} = \sqrt{\frac{R^2}{4L^2} =
\frac{1}{LC}}$ получим $i_{св} = e^{-\alpha t}(A e^{\beta t} +
Be^{-\beta t})$. Для вычисления неизвестных постоянных $A$ и $B$
необходимо использовать начальные условия.

Если предположить, что электромагнитная энергия в момент,
предшествующий короткому замыканию цепи, полностью сосредоточена в
ёмкости, то есть конденсатор $C$ при $t = (0_-)$ имеет напряжение
$U_0$, а ток в индуктивности $i(0_-) = 0$.

В соответствии с этим можно написать:
\begin{align*}
  U_C(0_+) &= U_0 \\
  i_{св}(0_+) &= i(0_-) = 0
\end{align*}

Используя условия, находим, что $B = -A$, откуда $i_{св} =
Ae^{-\alpha t}(e^{\beta t} - e^{\beta t})$. При $t = (0_+)$ для
выбранных условно положительных направлений тока и напряжения на
основании второго закона Кирхгофа имеем $U_C(0_+) + U_L(0_+) = 0$ или
с учётом $U_C(0_+) = U_0$.

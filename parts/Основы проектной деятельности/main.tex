\part[Основы проектной деятельности]{
Основы проектной деятельности \\
{\Large Зайцева Елена Михайловна}
}

\chapter{2025-02-06. Основные понятия}

\begin{definition}[Проектная деятельность]
	Совокупность действий, направленных на решение конкретной задачи в рамках
	проекта, ограниченного целевой установкой, сроками и достигнутыми
	результатами. В качестве результата может быть физический объект или какой-то
	информационный.
\end{definition}

\begin{definition}[Проект]
	Ограниченная по времени деятельность, направленная на решение определённой
	проблемы.
\end{definition}

\subsubsection{Этапы проектной деятельности}

\begin{enumerate}
	\item Определяется проблема, проходит поиск методов её решения;
	\item Составляется план работы над проектом;
	\item Реализация проекта;
	\item Представление проекта.
\end{enumerate}

На каждом этапы необходимо проводить поэтапную оценку.

\subsubsection{Правила работы над проектом}

\begin{enumerate}
	\item При работе в команде все члены команды равны;
	\item За полученный результат ответственность несут все члены команды;
\end{enumerate}

\subsubsection{Свойства проектов}

\begin{enumerate}
	\item Разовость;
	\item Уникальность;
	\item Инновационность;
	\item Результативность;
	\item Временная локация.
\end{enumerate}

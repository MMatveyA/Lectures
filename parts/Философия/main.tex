\part[Философия]{
Философия \\
{\Large Пилюшенко Александра Валерьевна}
}

\chapter{2025-02-04. Философия, её предмет и место в культуре}

\begin{definition}[Философия]
	Филио --- любовь, софия --- мудрость; то есть любовь к мудрости. Наука
	зарождается давно, ещё в Древнем мире. Как таковое познание мира начинается с
	философии.

	Философия --- изучает вопросы, ответы на которые не даются окончательно; там,
	где есть место для дискусии.
\end{definition}

Вопросы философии принипиально делятся на две большие группы:
\begin{itemize}
	\item Вопросы, которые исторически были философскими, а далее на них был дан
	      однозначный ответ в науке. Для современной философии это вообще не вопросы.
	\item Собственные философские вопросы, которые принипиально не поддаются
	      осмыслению в науке: вопросы о смысле жизни, долге и достоинстве,
	      сушествование человека и сущности.
\end{itemize}

\section{Разделы философии}

\subsection{Онтология}

\begin{definition}[Онтология]
	Учение о бытие.
\end{definition}

\subsubsection{Концепция космогенеза}

\begin{definition}[Космогенез]
	Космос --- вселенная, мир (дословно --- антоним хаоса); генез --- создание
	(активное создание), происхождение (пассивное создание).
\end{definition}

\begin{definition}[Креационизм]
	Онтологическая концепция, утверждающая, что вселенная или мир был создан
	сверх разумным существом (например, Богом).
\end{definition}

\begin{definition}[Теория Большого взрыва]
	Вселенная образовалась в результате взрыва тела максимальной температуры и
	максимальной плотности.
\end{definition}

\subsubsection{Проблема первоначала}

\begin{definition}[Первоначало]
	То, с чео всё началось; то, от чего всё произошло.
\end{definition}

Выделюятся две концепции:

\hrulefill

\begin{definition}[Материализм]
	Онтологическая концепция, утверждающая первичной реальностью материю, а идею
	вторичной или производной от материи.
\end{definition}

\begin{definition}[Материя]
	Это категория для обозначения объективной реальности, которая дана человеку в
	его ошушениях и существует независимо от них.

	Определение материи дал Владимир Ильич Ленин. Хотя оно было дано
	давно, оно считается эталонным. При этом, из-за развитий физики,
	оно несколько не точно.
\end{definition}

\hrulefill

\begin{definition}[Идеализм]
	Концепция, утверждающая, что первоначалом является идея, а материя вторична от
	идеи.
\end{definition}

\begin{definition}[Объективный идеализм]
	Концепция, утверждающая, что первоначало мира является объективная идея
	(например, Бог).
\end{definition}

\begin{definition}[Субъективный идеализм]
	Истинная реалность --- человечекое сознание. Нет вообще ничего другого, кроме
	сознания.
\end{definition}

\hrulefill

\subsubsection{Космология}

Что такое мир? Как он устроен?

\subsubsection{Общие законы существования мира}

\begin{definition}[Детерминизм]
	Это учение о прозрачности причинно--следственных связей. Появляется в 17-ом
	веке и является самым простым взглядом на мир.
\end{definition}

\begin{definition}[Индетерминизм]
	Онтологическая концепций, утверждающая не прямой характер
	причинно--следственных связей.
\end{definition}

\begin{definition}[Фатализм]
	Вера в рок, судьбу.
\end{definition}

\subsection{Гносиология}

\begin{definition}[Гносиология]
	Изучение методов изучения мира.
\end{definition}

\subsubsection{Субъект и объект познания}

Субъект это то, что получает познание (обычно человек). Объект это то, на что
направлено познание.

\subsubsection{Сущность и явление}

Сущность --- это смысл вещи; она всегда одна. Явление это отображение сущности в
органах чувство человека.

\subsubsection{Концепции истины}

Истина --- это искомый объект познания.

\begin{definition}[Классическая концепция]
	Соответствие знаний об объекте самому объекту.
\end{definition}

\begin{definition}[Конвенциальное]
	Истина как объект соглашения.
\end{definition}

\begin{definition}[Прагматическая]
	Истина, как практическая полезность.
\end{definition}
